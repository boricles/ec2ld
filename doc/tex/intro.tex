\section{Introduction}\label{sec:intro}
Virtualization is a computing technology that enables a single user to access multiple physical devices. Virtualization may also be used for running multiple applications on each server rather than just one; this in turn reduces the number of servers companies need to purchase and manage. It enables to consolidate servers and do more with less hardware. In the other hand, cloud computing offers scalable infrastructure and software off site, saving labor, hardware, and power costs. Financially, the cloud’s virtual resources are typically cheaper than dedicated physical resources connected to a personal computer or network. With cloud computing, the software programs are not running from the personal computer, but rather are stored on servers housed elsewhere and accessed via the Internet. \cite{Steinder_2008}.

Linked Data principles are being adopted by an increasing number of data providers, getting as a result a global data space on the Web containing billions of RDF triples \cite{Heath_Bizer_2011}. Moreover, Linked Data technologies are being using to share data covering a wide range of different topical domains, such as Media, Geographic, Government, Publications, Cross-domain, and Life sciences. However, there are no datasets that cover the Computer Science domain, which means that we, as computer scientists, are not taking the advantages of adopting Linked Data Principles to our own domain.

This paper aims at showing that is possible to adopt Linked Data Principles within the Computer Science domain, specifically in the virtualization and cloud computing fields. In this paper, we present the process that has been followed for the development of Linked Data applications that facilitate the search and discovery of virtual machine images in a systematic way. The rest of paper is organized as follows: Section \ref{sec:ec2} introduces the Amazon Elastic Compute Cloud (EC2) AMI, Section \ref{sec:process} explains the process we followed for the generation of linked
data, Section \ref{sec:apps} describes the linked data applications that we built, and finally, Section \ref{sec:conclusions} presents the conclusions and future work.







